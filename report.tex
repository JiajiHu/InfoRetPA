\documentclass[10pt]{article}
\usepackage{fullpage,enumerate,amsmath,amssymb,graphicx,setspace,epstopdf,float}

\begin{document}
\title{CS 276 Programming Assignment 1 Project Report}
\author{Jiaji Hu, Xuening Liu}
\date{}
\maketitle

\section{Program Structure}
\subsection{Code Structure}
The classes used in this programming assignment are as follows:

{\it Index} and {\it Query} are used to handle indexing and querying, respectively.

{\it BaseIndex} is the interface, and {\it BasicIndex}, {\it VBIndex} and {\it GammaIndex} do the encoding and decoding for writing to (reading from) output files.

{\it pairComparator} and {\it postingListComparator} are used to help sort the posting lists.

\subsection{Program}
The program does the following:

For indexing, we use Blocked sort-based indexing (BSBI). Globally, we maintain the term dictionary, document dictionary and the postings dictionary. The first two are used to save space in the index and for convenience in operations. The postings dictionary is used to quickly find the correct postings at query time.

In this assignment, the data is given to us already segmented into blocks. Next, we read the terms from the documents(files) of each block, generate a $<termID,docID>$ pair. All such pairs are then sorted by termID and then docID. From that, a postings list can be created for each termID. These are then written out to disk as intermediate results. In this assignment, we used three different encoding methods to do this, which will be discussed in detail in the next subsection.

In our implementation, we opted to include the length of the postings list (the number of docIDs) at the front of the posting list. By doing so, we no longer need to record the length of the lists in the postings dictionary. (Now, we can use compression encoding on that number, potentially saving some space in total.)

Finally, we merge the intermediate blocks into the final index. In this assignment, we perform binary merge on the blocks. The effect of the merge is that postings lists with the same termID will be combined. The algorithm to merge postings lists is quite similar to the algorithm for the intersection of postings lists, which we discussed and implemented in class.

To handle queries, we parse out the query terms, find their termIDs using the term dictionary, find their postings lists using the posting dictionary, and intersect the postings lists. For efficient lookup, the terms are ordered by the length of their postings lists. After the intersection, we translate the docIDs to document names, and output them as the result of the query.

\subsection{Encoding methods}
\begin{enumerate}[(a)]
\item Encoding for Basic algorithm

For task 1, we simply write out the 4-byte integers. In this scheme, every integer takes up 4 bytes.
\item Variable Byte Encoding

For task 2, there are two differences. Firstly, we use gap encoding for docIDs. This will make the numbers smaller. Secondly, we use Variable Byte encoding, where the first bit of each byte is used as a continuation bit. Now, small numbers take up much less space. However, doing this introduces extra steps for encoding/decoding, so it is slower than the uncompressed version. Therefore, there is a trade-off of less speed for better space when using variable bit encoding.
\item Gamma Encoding

Gamma encoding adapts the length of each integer on the bit level. It uses a unary code to represent a length, and then a binary code to represent the offset value. Using gamma encoding, we gain further performance in terms of compression. For example, only one bit is required to encode the integer `1'. Formally, gamma encoding achieves compression within a factor of optimal.

However, the encoding/decoding steps for gamma encoding can be more expensive, because there needs to be many bit-level operations, such as shifting and masking, in the encode-decode process. This leads to even more sacrifices in speed, and this is another trade-off situation. In general, gamma encoding gives the best compression, but is the slowest. The basic encoding is fastest, but uses the most disk space. VB encoding is somewhere in the middle.
\end{enumerate}
\section{Statistics}
\begin{table}[ht]
\begin{tabular}{| c | c | c | c | c |}
\hline
Algorithm & index time(s) & index size (MB) & index size - including dicts (MB) & average retrieval time (s)\\\hline
Basic & 67 & 58 & 72 & 1\\\hline
VB & 149 & 18 & 33 & 1\\\hline
Gamma & 178 & 13 & 29 & 1\\\hline
\end{tabular}
\end{table}
\section{Questions}
\begin{enumerate}[a)]
\item
The indexing process mainly consists of two steps: First, we process and sort the input data in blocks. Second, we merge the processed intermediate blocks into a single index.

With large blocks, the first step (the sorts) takes longer, but the second step can be faster (fewer merge passes). With smaller blocks, the first step takes less time, but more merge passes may be necessary, slowing down step 2. So there is a trade-off of time between step 1 and step 2 for different block sizes.

We note that the first part is asymptotically more expensive, but the second part can involve a large amount of extra disk I/Os for multiple passes with binary merge, which is very expensive in terms of time. Therefore, generally, it is a good idea to make the block size as large as the computer memory allows (we need to leave room for other things such as term dictionaries, etc.)

\item
With the current indexing program, the term dictionary could grow so large that it cannot fit in memory. If we are forced to move the term dictionary to disk, the extra I/Os would greatly affect the indexing time. Note that there is a similar issue with the document and postings dictionaries, but since we do not need to read from them very often, they pose less of a problem.

The following optimizations can be done:
\begin{enumerate}[(i)]
\item Use multi-way merge for merging intermediate blocks. Note that this can be done efficiently with heaps/priority queues. (Improves indexing time.)
\item Keep a cache in memory for the most popular queries, and/or extra intersected postings lists for frequent joint queries in index. (Improves retrieval time.)
\end{enumerate}
\item
\begin{enumerate}[(i)]
\item Instead of BSBI, use single-pass in-memory indexing(SPIMI). This is used to solve the scalability issue with oversized global dictionaries mentioned in the previous question.
\item The next step would be to use distributed indexing. Using a Map-Reduce architecture should solve scalability problems.
\end{enumerate}
\end{enumerate}
\end{document}